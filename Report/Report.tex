\documentclass[11pt, oneside]{article}   	% use "amsart" instead of "article" for AMSLaTeX format

\usepackage{array}
\setlength{\parindent}{4em}
\setlength{\parskip}{1em}
\renewcommand{\baselinestretch}{1.25}

\usepackage{geometry}                		% See geometry.pdf to learn the layout options. There are lots.
\geometry{letterpaper}                   		% ... or a4paper or a5paper or ... 
%\geometry{landscape}                		% Activate for rotated page geometry
%\usepackage[parfill]{parskip}    		% Activate to begin paragraphs with an empty line rather than an indent
\usepackage{graphicx}				% Use pdf, png, jpg, or eps§ with pdflatex; use eps in DVI mode
								% TeX will automatically convert eps --> pdf in pdflatex		
\usepackage{amssymb}
\usepackage{listings}
\usepackage{tikz-qtree}
\usepackage[document]{ragged2e}
\usepackage{graphicx}				% Use pdf, png, jpg, or eps§ with pdflatex; use eps in DVI mode	
\usepackage{amsmath}


%SetFonts

%SetFonts


\title{COMP9417 - Assignment 2}
\author{Avinash K. Gupta, Yuyang Shu, Maria Oei}
%\date{}							% Activate to display a given date or no date

\begin{document}
\maketitle
\newpage
\tableofcontents
\newpage
\section{Introduction}
\section{Introduction}
\par With over 100 millions of monthly visitor in Quora, it is inevitable that many people are asking similar questions. This has become an issue as user has to read through responses to many questions in order to find the best answer. The aim of this project is to implement algorithms to identify 2 similar questions which can help Quora in improving user experience by finding high quality answers to questions.

\par The 2 approaches used in this project were perceptron learning and LSTM. In the perceptron learning, we used 3 inputs which measured semantic similarity, word order and word overlaps between 2 sentences. In LSTM, we provide word-embedding vectors supplemented with synonymic information and calculates the manhattan distance between the vectors.

\par We will see that for our case the performance of the 2 models were quite similar. However, there are other research done on LSTM where it has been shown that LSTM has the potential to performs well with enough training and with careful selection of initial weight.


\section{Methodology}
In this project, we used wordnet as the main lexical database. Using wordnet, we were able to extract the part-of-speech tags for a word as well as their synset (synonym set). This played an important role in determining the similarity of 2 words as we would see later.

\subsection{Method 1 : Perceptron}

The first model was a a simple perceptron with 2 inputs which were based on [insert reference to paper here]. The inputs chosen measured the semantic similarity and the word order information of the 2 sentences. For each sentences pair in the dataset, both semantic similarity and word order score were calculated. The perceptron weights were then trained based on these measures.

\subsubsection{Words Similarity}
A \textbf{path length} between 2 words is the number of synsets we visit from one word to another. For example, in the figure below, to get from boy to girl we have to visit boy - male - person - female - girl. Therefore, the path length of 'boy' and 'girl' is 4. 'Person' is called the subsumer of 'boy' and 'girl'. If there are more than 1 path, we will consider the shortest path and the corresponding subsumer is called the \textbf{lowest subsumer}.


 \begin{figure}
\centering
\includegraphics[scale=0.7]{Synset_tree}
\caption{Synset Tree}
 \end{figure}

Let $l$ denote the shortest path and $h$ denote the depth of the lowest subsumer. The similarity between 2 words $w_1$ and $w_2$ is therefore measured by 
\begin{equation*}
s(w_1, w_2) = f_1(l) f_2(h)
\end{equation*}
where $\alpha, \beta \in [0,1]$ and 
\begin{align*}
	f_1(l)		&= e^{-\alpha l} \\
	f_2(h)	&= \frac{e^{\beta h} - e^{-\beta h}}{e^{\beta h} + e^{-\beta h}} \\
\end{align*}

\subsubsection{Semantic Similarity}
Let $T_1$ and $T_2$ be the 2 sentences and T is a set of distinct words in T1 and T2. For each sentence, we will calculate the vector $s_k$ which the same length as T. For each word $w_i$ in T, we assign 1 to the corresponding element in $s_k$ if $w_i$ is in $T_k$ and $\mu_i$ otherwise. $\mu_i$ is the similarity score between $w_i$ and the most similar word in $T_k$ calculated based on the similarity score above. Once $s_1$ and $s_2$ are calculated, the overall semantic similarity score is calculated by 
\begin{equation*}
	S_s = \frac{s_1. s_2}{||s_1||.||s2||}
\end{equation*}

\subsubsection{Word Order}
Similar to the semantic similarity, we calculate the vectors $r_1$ and $r_2$ for each of the sentences. For each word $w_i$ in T, we set the $i^{th}$ element in $r_k$ to equal to the position of $w_i$ in $T_k$. If $w_i$ is not in $T_k$ then we find the most similar word in $T_k$ and assign the position of that word instead.

For both word order and semantic similarity measure, we define a threshold for the case where $w_i$ is not in $T_k$. If the similarity score between $w_i$ and the most similar word is less than the threshold, 0 will be assigned instead.

For the word order, the overall score is calculated by 
\begin{equation*}
	S_r = 1 - \frac{||r_1 - r_2||}{||r_1 + r_2||}
\end{equation*}



%\pagebreak
\subsection{Method 2 : LSTM - RNN}
%
\subsubsection{Overview}
\par In this approach, we present  siamese adaption of Long Short-Term Memory(LSTM) network for labeled data contains pairs of variable-length sequences. This model is used to get the semantic similairty between the sentences using complex neural network. For this applications, we provide word-embedding vectors supplemented with synonymic information to the LSTMs, which use a fixed size vector to convert the syntactic meaning of the sentence. 

\subsubsection{Model}
\par It's a supervised learning model, where each data consists of pair of sequences 
$(x_1^{(a)},...,x_{n_a}^{(a)})$, $(x_1^{(b)},...,x_{n_b}^{(b)})$ of fixed size vectors along with a single label y(human labeled) for the pair. Note that sequences may be differenent length. These data will be pass to the model with a purpose of learning the semantics. 
 \begin{figure}[h]
    \centering
    \includegraphics[width=.75\textwidth]{lstm_image}
    \caption{lstm architecture}
\end{figure}
In the above diagram, there are two networks $LSTM_a$ and $LSTM_b$, each of them process a sentence in a given pair and predict whether  $LSTM_a=LSTM_b$ based on the similarity between the vectors. In each LSTM, the word vector is employ to the hidden layer and calculation is done and output is passed to the next hidden layer to remember the previous context and so on.
 \begin{figure}[h]
    \centering
    \includegraphics[width=.85\textwidth]{hidden_layer}
    \caption{repeating module of hidden layer}
\end{figure}
In above figure, the hidden layer which is core part of our neural network. It performs operation listed below on each word vector.
\begin{itemize}
\item The first sigmoid function is used to decide which information to keep and what to ignore
$$f_t=sigmoid(W_ix_t + U_ih_{t-1}+b_i)$$
\item Now it's time to update the old cell state into new cell state. The last step already decided what to do, in this step we just actually need to do it.
$$i_t=sigmoid(W_fx_t + U_fh_{t-1}+b_f)$$
$$c_t^`=\tanh(W_cx_t + U_ch_{t-1}+b_c)$$
\item In this step, we will multiply old state $c_{t-1}$ with $f_t$ to forget things we decided to forget earlier. Then we add to new candidate values
$$c_t=i_t \odot c_t^` + f_t \odot c_{t-1}$$
\item Final step defines what we actually need to output based on out filtered cell state and then $\tanh$(used to push values between -1 and 1) and multiply with sigmoid function to decide the parts we need to output.
$$o_t=sigmoid(W_ox_t + U_oh_{t-1}+b_o)$$
$$h_t=o_t \odot \tanh(c_t)$$
\end{itemize}

Above process are carried by both the LSTM and emploed to the Similarity Function given below, which calculates the Manhattan differnece between the ouputs.
$$g(h_{T_a}^{(a)},h_{T_b}^{(b)})=\exp(\textendash \parallel h_{T_a}^{(a)} \textendash h_{T_b}^{(b)}\parallel_1) \in [0,1]. $$



\subsubsection{Overview}
\par In this approach, we present  siamese adaption of Long Short-Term Memory(LSTM) network for labeled data contains pairs of variable-length sequences. This model is used to get the semantic similairty between the sentences using complex neural network. For this applications, we provide word-embedding vectors supplemented with synonymic information to the LSTMs, which use a fixed size vector to convert the syntactic meaning of the sentence. 

\subsubsection{Model}
\par It's a supervised learning model, where each data consists of pair of sequences 
$(x_1^{(a)},...,x_{n_a}^{(a)})$, $(x_1^{(b)},...,x_{n_b}^{(b)})$ of fixed size vectors along with a single label y(human labeled) for the pair. Note that sequences may be differenent length. These data will be pass to the model with a purpose of learning the semantics. 
 \begin{figure}[h]
    \centering
    \includegraphics[width=.75\textwidth]{lstm_image}
    \caption{lstm architecture}
\end{figure}
In the above diagram, there are two networks $LSTM_a$ and $LSTM_b$, each of them process a sentence in a given pair and predict whether  $LSTM_a=LSTM_b$ based on the similarity between the vectors. In each LSTM, the word vector is employ to the hidden layer and calculation is done and output is passed to the next hidden layer to remember the previous context and so on.
 \begin{figure}[h]
    \centering
    \includegraphics[width=.85\textwidth]{hidden_layer}
    \caption{repeating module of hidden layer}
\end{figure}
In above figure, the hidden layer which is core part of our neural network. It performs operation listed below on each word vector.
\begin{itemize}
\item The first sigmoid function is used to decide which information to keep and what to ignore
$$f_t=sigmoid(W_ix_t + U_ih_{t-1}+b_i)$$
\item Now it's time to update the old cell state into new cell state. The last step already decided what to do, in this step we just actually need to do it.
$$i_t=sigmoid(W_fx_t + U_fh_{t-1}+b_f)$$
$$c_t^`=\tanh(W_cx_t + U_ch_{t-1}+b_c)$$
\item In this step, we will multiply old state $c_{t-1}$ with $f_t$ to forget things we decided to forget earlier. Then we add to new candidate values
$$c_t=i_t \odot c_t^` + f_t \odot c_{t-1}$$
\item Final step defines what we actually need to output based on out filtered cell state and then $\tanh$(used to push values between -1 and 1) and multiply with sigmoid function to decide the parts we need to output.
$$o_t=sigmoid(W_ox_t + U_oh_{t-1}+b_o)$$
$$h_t=o_t \odot \tanh(c_t)$$
\end{itemize}

Above process are carried by both the LSTM and emploed to the Similarity Function given below, which calculates the Manhattan differnece between the ouputs.
$$g(h_{T_a}^{(a)},h_{T_b}^{(b)})=\exp(\textendash \parallel h_{T_a}^{(a)} \textendash h_{T_b}^{(b)}\parallel_1) \in [0,1]. $$
\section{Dependencies}

\begin{enumerate}
\item   Python packages

\begin{itemize}
\item nltk
\item numpy
\item Scipy
\item gensim
\item Theano
\item cython
\end{itemize}
\item Tested System Configuration
\begin{itemize}

\item intel i5 6200u
\item Intel HD Graphics 520
\item 8 GB Ram
\item 128GB SSD

\end{itemize}

\end{enumerate}

\section{Result}
%\section{Result \& Discussion}
\justify
In the perceptron model, we see the naive precision ((true positive + true negative) / total data points) for the best performance is only around 0.50, i.e. not much better than random guess.

The first challenge we face is, what should our vocabulary be, if we were to project the vocabulary into a vector space ourselves. One way is to use all words in the training set, but since embedding requires huge amounts of data to be accurate, this can't work well. Another way could be, to retrieve all questions on Quora, and use the words in all those questions as the vocabulary. The problem of this method is, we would actually be peeking into the testing set, since inevitably the testing set would be from Quora. A third method would be, to use some third party source to construct the vocabulary and the corresponding word vectors.

We believe the third method is the best, and hence we use the Google News vectors (https://drive.google.com/file/d/0B7XkCwpI5KDYNlNUTTlSS21pQmM) as our embedding.

The second challenge is that, due to the computational complexity and hardware dependencies of the model, specifically the requirements of the Theano package, we were not able to replicate the training of the LSTM model ourselves before the deadline of the project. Instead, we used a pre-trained model to do the prediction, namely the Siamese LSTM (https://github.com/aditya1503/Siamese-LSTM) model trained based on the sentences involving compositional knowledge dataset (Marelli et al 2014) as a proxy. We feed the question pairs into the Siamese model, and it would produce the similarity measure between them. We then run a simple test to find the threshold beyond which we should predict duplicate. Based on the test, setting the threshold at 0.19 can give us a best f1 score of 0.59, and setting the threshold at 0.34 we can have a best "naive" prediction precision of 66%.

If we were to train the LSTM RNN ourselves, we would feed it with our labelled data, in the form of (question 1, question2, label) where label is a boolean indicating whether the two questions are duplicate or not, and the test data would be in the form of (question 1, question2). This way, since the model would be directly predicting whether the questions are duplicate or not instead of their semantic similarity, the model performance would be much better.




\section{Discussion}
\subsection{Conclusion}
 \subsection{Limitation \& Improvement}

\newpage

\medskip
 
\begin{thebibliography}{9}

\bibitem{colah's blog} 
colah's blog.
\textit{Understanding LSTM Networks}.
\\\texttt{http://colah.github.io/posts/2015-08-Understanding-LSTMs/}

\bibitem{AAAI Publications, Thirtieth AAAI Conference on Artificial Intelligence} 
AAAI Publications, Thirtieth AAAI Conference on Artificial Intelligence.
\textit{Siamese Recurrent Architectures for Learning Sentence Similarity}.
\\\texttt{https://www.aaai.org/ocs/index.php/AAAI/AAAI16/paper/viewFile/12195/12023}


\end{thebibliography}


\end{document}  